\documentclass[12pt,a4paper,fleqn]{article}
\usepackage[utf8]{inputenc}
\usepackage{amsmath}
\usepackage{amsthm}
\usepackage{amsfonts}
\usepackage{amssymb}
\usepackage[overload]{empheq}
\usepackage{fullpage}
\usepackage{multicol}
\usepackage{enumerate}
\usepackage{graphicx}
\usepackage[margin=10pt,font=small,labelfont=bf, labelsep=endash, justification=centering]{caption}

\title{Model for the truck-and-freighter routing problem}
\author{Dorian Dumez}

\begin{document}

\maketitle

Les constantes et les données du problèmes sont notées :
\begin{itemize}
\item Le nœuds sont :
\begin{itemize}
\item le nœud $1$ est le dépôt 
\item $J = \{1,...,n\}$ est l'ensemble des clients, où :
\begin{itemize}
\item $J_S \subset J$ est les clients ne pouvant être livré que par le petit
\item $J_B \subset J$ est les clients ne pouvant être livré que par le gros
\item $J_R = J \backslash (J_B \cup J_S)$ mes clients pouvant être livré par les 2
\end{itemize}
\item $P$ est l'ensemble des parking
\item $V = \{0, n+1\} \cup J \cup P$, $E = V^2$
\end{itemize}
\item les demandes et capacités sont :
\begin{itemize}
\item $\forall j \in J \text{ : } q_j$ la demande du client j
\item $Q^B = \infty$ la capacité du gros
\item $Q^S$ la capacité du petit
\end{itemize}
\item les coûts sont :
\begin{itemize}
\item $\forall (i,j) \in E \text{ : } c^B_{ij}$ le coût du trajet $(i,j)$ avec le gros (ou les deux ensembles)
\item $\forall (i,j) \in E \text{ : } c^S_{ij}$ le coût du trajet $(i,j)$ avec le petit
\end{itemize} 
\item les trucs de restrictions temporelles sont :
\begin{itemize}
\item $\forall (i,j) \in E \text{ : } t_{ij}$ le temps de trajet de $(i,j)$
\item $\forall j \in J \text{ : } [a_j,b_j]$ la fenêtre de temps du client j
\item $e$ la date finale à laquelle tout le monde doit être rentré
\item $s$ la durée du service chez un client
\item $ta$ le temps d'assemblage des 2 véhicules (fixé au même temps que la durée d'un service chez un client)
\item $r$ le ratio de vitesse entre les 2 véhicules
\end{itemize}
\end{itemize}

\clearpage
Les variables sont :
\begin{itemize}
\item pour la gestion du tour du gros :
\begin{itemize}
\item $\forall (i,j) \in E \text{ : } X_{ij} \in \{0,1\}$ le gros vas de $i$ à $j$
\item $\forall i \in V \text{ : } S_i \in \mathbf{R}^+$ le gros commence à servir $i$ à cette date ou à être sur le parking. $S_1$ est un cas particulier, il représente la date de retour au dépôt
\end{itemize}
\item pour la gestion de la k-ieme tournée du petit($k \in ]K]$, $K$ max de tournées du petit):
\begin{itemize}
\item $\forall (i,j) \in E \text{ : } x_{ij}^k \in \{0,1\}$ le petit vas de $i$ à $j$
\item $\forall i \in V \text{ : } s_i^k \in \mathbf{R}^+$ le petit commence à servir $i$ à cette date ou à être sur le parking
\end{itemize}
\item pour la gestion de la synchronisation petit/gros :
\begin{itemize}
\item $\forall k \in ]K] \text{ : } \forall i \in P \text{ : } d^k_i \in \{0,1\}$ la tourné $k$ du petit démarre en $i$ 
\item $\forall k \in ]K] \text{ : } \forall i \in P \text{ : } f^k_i \in \{0,1\}$ la tourné $k$ du petit fini en $i$
\end{itemize}
\end{itemize}

Les contraintes sont :
\begin{itemize}
\item \ref{tousservis} : tous les clients sont servis
\item \ref{fenetrerestant1}, \ref{fenetrerestant2}, \ref{fenetregros}, \ref{fenetrepetit} : les fenêtre de temps des livraisons sont respectées
\item \ref{sequentialitegros1}, \ref{sequentialitegros2}, \ref{sequentialitegros3} : les contrainte de sequentialité du gros dans les divers cas
\item \ref{sequentialitepetit1}, \ref{sequentialitepetit2} : contrainte de sequentalité du petit
\item \ref{findejournee} : la date maximale de retour au dépôt
\item \ref{partirdudepot} et \ref{reveniraudepot} : respectivement qu'il faut partir du dépôt et y revenir
\item \ref{separationvalide}, \ref{partirdugros} et \ref{mergevalide}, \ref{reveniraugros} : respectivement que le petit doit partir et revenir au gros
\item \ref{datedebutpetit} et \ref{datedefinpetit} : respectivement que les séparations et fusion se font bien au même moments
\item \ref{flotgros} et \ref{flotpetit} : contrainte de flot respectivement sur les déplacements du gros et du petit
\item \ref{capa} : contrainte de capacité du petit 
\end{itemize}


\begin{align}
& \text{min } && \sum \limits_{(i,j) \in V} (c^B_{ij}X_{ij} + \sum \limits_{k \in ]K]}( c^S_{ij}x^k_{ij})) \\[10pt]
& \text{sc }  && \forall j \in J \text{ : } \sum \limits_{i \in V}( X_{ij} + \sum \limits_{k \in ]K]} x^k_{ij} ) = 1 \label{tousservis} \\
& && \forall i \in J_R \text{ : } \forall k \in ]K] \text{ : } a_i \leqslant s_i^k \leqslant b_i \label{fenetrerestant1} \\
& && \forall i \in J_R \text{ : } a_i \leqslant S_i \leqslant b_i \label{fenetrerestant2} \\[30pt]
& && \sum \limits_{j \in V} X_{1,j} = 1 \label{partirdudepot} \\
& && \forall h \in J \cup P \text{ : } \sum \limits_{i \in V} (X_{ih}) - \sum \limits_{j \in V} (X_{hj}) = 0 \label{flotgros} \\
& && \sum \limits_{i \in V} X_{i,1} = 1 \label{reveniraudepot} \\
& && \forall i \in J \text{ : } \forall j \in V \text{ : } S_i + t_{ij} + s - M(1-X_{ij}) \leqslant S_j \label{sequentialitegros1} \\
& && \forall i \in P \text{ : } \forall j \in V \text{ : } S_i + t_{ij} + a - M(1-X_{ij}) \leqslant S_j \label{sequentialitegros2} \\
& && \forall j \in V \text{ : } 0 + t_{1j} - M(1-X_{1j}) \leqslant S_j \label{sequentialitegros3} \\
& && S_{1} \leqslant e \label{findejournee} \\
& && \forall i \in J_B \text{ : } a_i \leqslant S_i \leqslant b_i \label{fenetregros} \\[30pt]
& && \forall k \in ]K] \text{ : } \forall j \in P \text{ : } d^k_j \leqslant \sum \limits_{i \in V} X_{ij} \label{separationvalide} \\
& && \forall k \in ]K] \text{ : } \forall j \in P \text{ : } f^k_j \leqslant \sum \limits_{i \in V} X_{ij} \label{mergevalide} \\
& && \forall k \in ]K] \text{ : } \forall i \in P \text{ : } \sum \limits_{j \in V} x^k_{ij} = d^k_i \label{partirdugros} \\
& && \forall k \in ]K] \text{ : } \forall h \in J \text{ : } \sum \limits_{i \in V} (x_{ih}^k) - \sum \limits_{j \in V} (x_{hj}^k) = 0 \label{flotpetit} \\
& && \forall k \in ]K] \text{ : } \forall j \in P \text{ : } \sum \limits_{i \in V} x^k_{ij} = f^k_j \label{reveniraugros} \\
& && \forall k \in ]K] \text{ : } \forall i \in J \text{ : } \forall j \in V \text{ : } s_i^k + r*t_{ij} + s - M(1-x_{ij}^k) \leqslant s_j^k \label{sequentialitepetit1} \\
& && \forall k \in ]K] \text{ : } \forall i \in P \text{ : } \forall j \in V \text{ : } s_i^k + r*t_{ij} + a - M(1-x_{ij}^k) \leqslant s_j^k \label{sequentialitepetit2} \\
& && \forall k \in ]K] \text{ : } \forall i \in P \text{ : } s_i^k \geqslant S_i - M.(1 - d_i^k) \label{datedebutpetit} \\
& && \forall k \in ]K] \text{ : } \forall i \in P \text{ : } \forall j \in V \text{ : } s_i^k + a + t_{ij} \leqslant S_j + M.(1 - f^k_i) + M.(1 - X_{ij}) \label{datedefinpetit} \\
& && \forall k \in ]K] \text{ : } \forall i \in J_S \text{ : } a_i \leqslant s_i^k \leqslant b_i \label{fenetrepetit} \\
& && \forall k \in ]K] \text{ : } \sum \limits_{j \in J} (q_j \sum \limits_{i \in V} x^k_{ij}) \leqslant Q_S  \label{capa} 
\end{align}

\clearpage

Les variables que l'on peut/doit fixer sont :
\begin{itemize}
\item $\forall k \in ]K] \text{ : } \forall i \in V : x^k_{i0}, x^k_{0i}, x^k_{i,n+1}, x^k_{n+1,i}, s^k_0, s^k_{n+1} = 0$ car le petit n'interagit pas avec le dépôt
\item $S_0 = 0$ et $S_{n+1} = e$ car il existe toujours une telle solution optimale
\item $\forall i \in V \text{ : } X_{i0}, X_{n+1,i} = 0$ car on ne fait que partir de $0$ et que revenir en $n+1$
\item $\forall k \in ]K] \text{ : } \forall p,p' \in P : x^k_{p,p'} = 0$ car un tel sous circuit est inutile
\item $\forall k \in ]K] \text{ : } \forall i \in V : x^k_{ii}, X_{ii} = 0$ car cela ne sert à rien
\item $\forall k \in ]K] \text{ : } \forall i \in J_B \text{ : } \forall j \in V \text{ : } x^k_{ij}, x^k_{ji} = 0$ car le petit n'a rien à faire chez les gros clients
\item $\forall i \in J_S \text{ : } \forall j \in V \text{ : } X_{ij}, X_{ji} = 0$ car le gros n'a rien à faire chez les petits clients
\item $X_{0,n+1} = 0$ si le problème n'est pas trivial
\end{itemize}

Les grandes valeurs à fixer :
\begin{itemize}
\item $M = e$ car c'est le plus grand temps possible
\item $K = |J_S|$ car on ne peut avoir besoin de plus de sous circuits
\item le nombre de duplication des parkings : K
\end{itemize}

\clearpage

Les contraintes redondantes que l'on peut essayer sont :
\begin{itemize}
\item bla
\end{itemize}


Les symétries que l'on peut casser sont :
\begin{itemize}
\item pour les duplications de parkings : forcer à prendre en premiers les première duplications, et dans l'ordre $\Rightarrow$ contraintes sur les $S_i$ et les $X_{ij}$
\item prendre les premiers sous tours et dans l'ordre :
\begin{align}
& \forall k \in ]K-1] \text{ : } \forall i,j \in P : d^k_i \geqslant d^{k+1}_j \label{symetriesoustours1} \\
& \forall k \in ]K-1] \text{ : } \forall i,j \in P : s^k_i \leqslant s^{k+1}_j \label{symetriesoustours2}
\end{align}
\end{itemize}

\end{document}